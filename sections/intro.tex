\section{Introduction}
\label{sec:intro}

Systems-on-chip (SoCs) that integrate a heterogeneous mix of complex IP blocks
can utilize on-chip shared memory to improve performance and simplify
programming \cite{spandex, cohort, li2023duet, mantovani_cases16}.  Like that
of traditional homogeneous architectures, the memory hierarchy of heterogeneous
SoCs may feature multiple levels of caches, orchestrated by a cache coherence
protocol. For example, Giri et al. propose a cache hierarchy for heterogeneous
SoCs built on top of a network-on-chip (NoC) \cite{giri18}.  Their cache
hierarchy supports multicore execution of processor cores and enables multiple
different coherence modes for hardware accelerators.  Utilization of these
modes can improve execution time and reduce off-chip memory acceses for a
variety of different workloads \cite{giri_ieeemicro18, zuckerman_micro21}.

\par ESP is an open-source platform for hetergeneous SoC design that simplifies
the integration of new IP blocks within a scalable SoC architecture and allows
for rapid prototyping of full systems \cite{mantovani_iccad20}. ESP's cache
hierarchy implements the protocol proposed by Giri et al -- an extended MESI
directory-based protocol.  The L1 of open-source processors (that potentially
implement different ISAs) interfaces to the ESP L2. The last-level cache (LLC)
and directory controller not only orchestrate coherence across these L2 caches
but also enable accelerators without a private cache to leverage the cache
hierarchy to access on-chip shared data.

\par In this work, we present an improved version of the ESP LLC. While the
original version is implemented with a multi-cycle datapath, the improved
version features a pipelined datapath, allowing for significantly greater
throughput. Since the last-level cache is the main synchronization point for
the various IPs throughout an SoC, this can give significant performance
improvements when there is a high density of requests at the LLC. In
particular, we show how the pipelined datapath can improve the performance of
accelerators that acceess their data directly from the LLC.
